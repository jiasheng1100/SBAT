\documentclass[a4paper]{article}
\usepackage[T1]{fontenc}
\usepackage[utf8]{inputenc}
\usepackage{natbib}

\usepackage[margin=1.25in]{geometry}
\usepackage{hyperref} % links
\usepackage{parskip} % proper paragraphs, no indentation
\usepackage{pdfpages} % add non-plagiarism statement

% For showing the month + year only as the date:
\usepackage[en-GB]{datetime2}
\DTMlangsetup{showdayofmonth=false}

\usepackage{xcolor}
\usepackage{blindtext} % for adding the "lorem ipsum" filler sections. delete this once you start writing your thesis.

\usepackage{graphicx}
\graphicspath{{./images/}}
\usepackage{float}

% This is used in the PDF meta data
\title{SBAT: A Simple Browser-based Annotation Tool with Interactive UI and Github Support}
\author{Jia Sheng}
\date{\today}

\begin{document}

% The actual title page of your thesis:
\begin{titlepage}
\begin{center}

\hrule
\vspace{0.6cm}
{\bfseries\LARGE
SBAT: A Simple Browser-based Annotation Tool with Interative UI and Github Support
}\\[1cm]
\hrule
\vspace*{.05\textheight}
 
\begin{minipage}[t]{0.49\textwidth}
\begin{flushleft} 
{\large
\textit{Author}\\
Jia Sheng}\\
\href{mailto:jia.sheng@student.uni-tuebingen.de}{\textit{jia.sheng@student.uni-tuebingen.de}}\\
\end{flushleft}
\end{minipage}
\begin{minipage}[t]{0.49\textwidth}
\begin{flushright}
{\large
\textit{Supervisor}\\
Çagri Çöltekin}\\
\href{mailto:ccoltekin@sfs.uni-tuebingen.de}{\textit{ccoltekin@sfs.uni-tuebingen.de}}\\
\end{flushright}
\end{minipage}\\

\vfill

\vspace*{.1\textheight}

{\large Seminar für Sprachwissenschaft\\
Eberhard Karls Universität Tübingen

\vspace{1em}
\today}
\end{center}
\end{titlepage}

% No page numbering for the abstract, the anti-plagiarism statement and the table of contents.
\pagenumbering{gobble}

%% Uncomment once you want to add the anti-plagiarism statement file
% \newpage
% \includepdf[pages=-]{name-of-anti-plagiarism-statement-file}

\newpage
%\tableofcontents
%\listoftables
%\listoffigures
\newpage

\pagenumbering{arabic}

\section{Introduction}
This paper introduces a SBAT, a light-weighted and easy-to-use web annotation tool. Although there are already various apps developed for the purpose of text annotation, they often required to be hosted on a server, which increases the threshold of usage. In comparison, SBAT is developed to be used in a web UI without need of registration or a dedicated server, while still keeping the necessary functions of the heavier annotation tools.
\begin{enumerate}
\item 
Motivation: \\
Easy to use for smaller projects, no budget for a server\\
large annotation projects involve many annotators\\
a centrally managed tool needed\\
\item limitation of current annotation tool:\\ 
needs to be hosted on a server\\
offline version needs installation\\
heavy\\
\item introduce SBAT, a browser-based tool for text annotation, can be combined well with github
\end{enumerate}
\section{Related Work}
\begin{enumerate}
\item brat
\item brat-frontend-editor
\item sbat old version by Charlotte
\item ud-annotatrix
\item inception
\end{enumerate}
\section{Features}
\begin{enumerate}
\item High-quality Annotation Visualisation and intuitive annotation interface from Brat
\item web interface, no need to register
\item import and export annotations through ui from github
\end{enumerate}
\section{Implementation}
\begin{enumerate}
\item annotation ui from brat
\item static web app hosted on github pages, no need of server, no server-side configuration, light-weighted
\item github API for reading and writing annotation files
\end{enumerate}

\section{Conclusion}
SBAT, a light-weight and easy-to-use tool for text annotation projects.\\ 
Annotation UI based on Brat, but made completely online, and enables user to import from / export to their own github repo.



%\begin{figure}[H]
%\centering
%\includegraphics[scale=0.2]{rag.png}
%\caption{Retrieval Augmented Generation}
%Source: https://techcommunity.microsoft.com/t5/azure-ai-services-blog/revolutionize-your-enterprise-data-with-chatgpt-next-gen-apps-w/ba-p/3762087
%\end{figure}



%\begin{figure}[H]
%\centering
%\includegraphics[scale=0.6]{flow_chart.png}
%\caption{Flow chart}
%\end{figure}

%\begin{enumerate}
%\item 
%\item 
%\item 
%\end{enumerate}


%\bibliography{bibliography}
%\bibliographystyle{chicago}

\end{document}
