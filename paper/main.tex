\documentclass[12ptm a4paper]{article}
\usepackage[T1]{fontenc}
\usepackage[utf8]{inputenc}
\usepackage{natbib}
\usepackage{setspace}
\usepackage[margin=1.25in]{geometry}
\usepackage{hyperref} % links
\usepackage{parskip} % proper paragraphs, no indentation
\usepackage{pdfpages} % add non-plagiarism statement

% For showing the month + year only as the date:
\usepackage[en-GB]{datetime2}
\DTMlangsetup{showdayofmonth=false}

\usepackage{xcolor}
\usepackage{blindtext} % for adding the "lorem ipsum" filler sections. delete this once you start writing your thesis.

\usepackage{graphicx}
\graphicspath{{./images/}}
\usepackage{float}

\linespread{1.5}

% This is used in the PDF meta data
\title{SBAT: A Simple Browser-based Annotation Tool with Interactive Visualization and Github Support}
\author{Jia Sheng}
\date{\today}

\begin{document}

% The actual title page of your thesis:
\begin{titlepage}
\begin{center}

\hrule
\vspace{0.6cm}
{\bfseries\LARGE
SBAT: A Simple Browser-Based Annotation Tool with Interative Visualization and Github Support
}\\[1cm]
\hrule
\vspace*{.05\textheight}
 
\begin{minipage}[t]{0.49\textwidth}
\begin{flushleft}
{\large
\textit{Author}\\
Jia Sheng}\\
\href{mailto:jia.sheng@student.uni-tuebingen.de}{\textit{jia.sheng@student.uni-tuebingen.de}}\\
\end{flushleft}
\end{minipage}
\begin{minipage}[t]{0.49\textwidth}
\begin{flushright}
{\large
\textit{Supervisor}\\
Çagri Çöltekin}\\
\href{mailto:ccoltekin@sfs.uni-tuebingen.de}{\textit{ccoltekin@sfs.uni-tuebingen.de}}\\
\end{flushright}
\end{minipage}\\

\vfill

A thesis submitted in partial fulfilment\\
of the requirements for the degree of\\[2mm]
{\large Bachelor of Arts}\\
in\\[1mm]
{\large International Studies in Computational Linguistics}

\vspace*{.1\textheight}

{\large Seminar für Sprachwissenschaft\\
Eberhard Karls Universität Tübingen

\vspace{1em}
\today}
\end{center}
\end{titlepage}

% No page numbering for the abstract, the anti-plagiarism statement and the table of contents.
\pagenumbering{gobble}

%% Uncomment once you want to add the anti-plagiarism statement file
% \newpage
% \includepdf[pages=-]{name-of-anti-plagiarism-statement-file}
\newpage
\section*{\centering{Abstract}}
In computational linguistics, annotating linguistic data is a crucial step for many studies, and many tools can be used to facilitate creation of annotated data. This paper introduces SBAT, a lightweight and easy-to-use web annotation tool. It supports annotation tasks that require attaching labels to text data, such as POS tagging or labeling for named entity recognition. Among the various applications previously developed for the purpose of text annotation, many require a dedicated server or a local installation, which complicates the usage. In contrary, SBAT is designed to be an online annotation tool that is easy to access and simple to use. It is built on the Brat annotation interface, but is completely browser-based and server-free, while preserving most of the key functionalities. Additionally, it includes support for integration with Github's file management system.
\newpage
\tableofcontents
\newpage
%\listoftables
\listoffigures
\newpage
\pagenumbering{arabic}

\section{Introduction}
Annotated data serves as the foundation for many linguistic studies, based on which it is possible to examine patterns, study linguistic phenomena, train statistical models or testify linguistic theories. And among all types of linguistic data, text annotation is the most prevalent. Text annotation refers to the labeling or classifying of text data, which is then used for various purposes. For instance, the text data with named entities tags can be used to train a NER (named entity recognition) model for Natural Language Processing. But annotating is not limited to named entities, tagging semantic roles, dependencies, parts of speech, sentiments, relations and so on are all considered text annotations in common sense. Moreover, although automatic annotations by the computer achieved good performance in some cases, most annotation tasks still require the annotation or validation of human annotators, highlighting the necessity of annotation tools.\\
\\
For manual or computer-assisted creations of annotated data, a variety of applications have been developed over the years, each with a specific focus or for a specific annotation scheme. These tools can recognize spans of characters and their links, and enable the user to associate labels to those spans and links \citep{inbook}. Notable examples include BRAT for text span annotation and relation annotation, UD Annotatrix for Universal Dependencies annotation, and INCEpTION for Semantic annotation. These tools not only provide an editor-like user interface for annotating operations, but many also enable suggestions generated by machine-learning models to speed up the annotation creation and increase the accuracy. Designed for usage in annotation projects, most of them provide user management functionalities through user registration and role assignment, in order to facilitate collaboration of annotators in the annotation process. In terms of usage, many of those tools are dependent on Java or Python libraries, and therefore require installation of Java or Python prior to use. Some of them have a browser-based frontend, while others are desktop applications.\\
\\
While those existing annotation tools provide various solutions across different schemes for the conduction of annotation projects, they often require complicated configuration on either local devices or the server side \citep{yang2018yeddalightweightcollaborativetext}, which is not ideal for smaller annotation projects with limited time and low budget. Furthermore, most of them do not have an efficient version control system, which means keeping track of changes or reverting back to previous versions is either impossible or requires a lot of effort. Taking those limitations into account, SBAT\footnote{Github repository: https://github.com/jiasheng1100/SBAT} is designed to be a browser-only, lightweight text span annotation tool, with integrated support for reading and writing files from and to Github repositories. By making SBAT browser-based, its only software dependency is a web browser, which is already available on every modern computer. And its support for Github repository operations opens the door to the application of Github's version management functionalities in the annotation project.\\
\\
The rest of this paper is therefore divided into four sections, all dedicated to the introduction and discussion of annotation tools and SBAT. The second chapter deals with the related work in the field, presenting a few representative annotation tools developed over the years, along with an overview of their functionalities, advantages and limitations. The third chapter introduces the main features of SBAT, such as its interactive user interface and its support for Github, backed up with detailed instructions and illustrations. Following that, the fourth chapter talks about the implementation details in the development of SBAT. Finally, this paper ends with concluding remarks about SBAT, pointing out its strength and limitations.


\newpage
\section{Related Work}
Among the annotation tools, BRAT \citep{stenetorp-etal-2012-brat} is a powerful scheme-neutral annotation tool with an intuitive user interface. It can be used for tasks including but not limited to POS tagging, named entity recognition, semantic role labeling, dependency and verb-frame annotations, and enables user-defined constraints-checking. Moreover, it provides a high-quality visualization of annotations by its browser-based UI component implemented in XHTML, Scalable Vector Graphics (SVG) and JavaScript. Last but not least, Brat integrats a ML-based semantic class disambiguator that outputs multiple annotations with their possibility estimates, which has been proven to increase the efficiency of annotators in several annotation projects.\\
\\
Despite the success of BRAT, it is not the simplest tool to start annotating with. A CGI-capable server is required by its installation, in order to combine all user modifications in real-time with the stored data. That is where brat-frontend-editor \citep{brat-frontend-editor}, a forked project of Brat, differentiates itself from the original project. Brat-frontend-editor is a standalone browser version of Brat that keeps some of its essential editing functions but removes its server-side code. It can be imported as a module in vanilla JavaScript, Angular or React. This project is since 2018 no longer maintained, so many dependencies have broken, and it no longer keeps up with the new updates of the original Brat. However, its transformation of server-side operations into browser-side operations serves as an important reference in the development of SBAT.\\
\\
While BRAT serves as a general-purpose annotation tool, more recent tools target specific annotation schemes. UD Annotatrix \citep{tyers-etal-2017-ud}, for example, is specifically designed for annotations of universal dependencies. It accepts several input formats including plain text and CoNNL-U, and offers graphical editing functions. Unlike BRAT, it offers a stand-alone module written in JavaScript with locally saved dependencies, which stores imported corpora in localStorage and allows offline usage. Otherwise, there is also a server module, which saves corpora on the server when enabled.\\
\\
Another domain-specific annotation platform is INCEpTION \citep{klie-etal-2018-inception}, which specializes in semantic annotation. One of its characteristics is the extensive adoption of machine-learning models to assist the annotation process. It not only utilizes recommenders based on trained models to provide users with suggestions for possible labels, but also provides an active learning mode where user feedback can be used to further improve the quality of those suggestions. It also has a comprehensive user management system, where the admin can create multiple user accounts for the project and assign them different roles. In terms of the software itself, INCEpTION is a Java application with Spring Boot backend and a web user interface, so its usage requires the installation of Java as well as the software.\\
\\
Lastly, the previous version of SBAT \citep{SBAT} serves as one of the foundations for the current version. The previous version shares the same goal: to create a simple but functional annotation tool that operates entirely in the browser, without any server-side configuration. It is implemented using HTML and JavaScript, and can be used to attach user-defined labels to whole texts. It also allows the local import and export of annotation files. Building on that, the current version addresses the limitations of the previous one by adding functionalities such as span annotations, annotation visualization, and Github support.
\newpage
\section{Features}
\subsection{Interactive User Interface}
Built upon BRAT, the annotation editor of SBAT is responsive, interactive and intuitive. To add an annotation, one can simply double-click or drag to select a span of characters with his mouse. A pop-up window will then appear [\hyperref[figure1]{Figure 1}], with a list of options of labels that the user has pre-defined in the config file. The user can then select the corresponding label in the pop-up menu, click the "OK" button, and the selected label is added as an annotation. In the editor, all annotated spans are highlighted with different color, and the appended labels are visualized in tags above the spans [\hyperref[figure2]{Figure 2}]. If there is any unclarity, users can also add their comments to the span, which will become visible with a mouse hover. Moreover, the label options and their styles can be defined by users in the JSON config file, and the size and display of the editor can be adjusted by the option button. Besides the main annotation editor, there is also an area for Github authentication as well as for the the annotation exportation, all guided with clear instructions.
\begin{figure}[H]
{\centering
\includegraphics[scale=0.55]{addNewAnnotation.png}
\label{figure1}
\caption{Pop-Up Menu for New Annotation}
}
{\setstretch{1.0}\scriptsize Appears when double-clicking or dragging to select a span of text in the editor. The displayed options are defined in the config file, in this case two entity types, "Person" and "Object".\par} 
\end{figure}
\subsection{Github Support}
SBAT enables the users to import files from or export annotations to the Github repositories that they have access to. To do so, one only has to enter his Github Personal Access Token, and can then select the branch and the filename to open the file in the annotation editor. To export annotations, the user can optionally enter a commit message in the input field and then click the "commit" button. In the whole process, if any error occurs in any step, the user will be prompted with the error message with hints to fix it, or, in terms of successful commit, also a message that indicates success. And by making use of Github commit, the versions of annotations can be well documented and managed.\\
\\
For importing files, all file formats are supported, and they will be treated as plain text files, i.e. the complete content will be displayed as plain text in the editor for annotation. The only exception is the JSON file which has been previously exported from SBAT. These JSON files have standard attributes including text, tokens, entities, and so on, and when imported in SBAT, their attribute values will be parsed, and the annotations which have been previously saved in the file will be visualized in the editor [\hyperref[figure2]{Figure 2}]. And after the user hits the commit button, a JSON file with the original text and its annotation meta data will be committed to the same Github repository from which the file was originally imported. One example of such SBAT-generated json files can be found in the "example data" folder.

\begin{figure}[H]
{\centering
\includegraphics[scale=0.55]{visualization.png}
\label{figure2}
\caption{Visualization of Annotations}
}
{\setstretch{1.0}\scriptsize Annotations can be exported into a JSON file, and when loading this file in SBAT, all previous annotations will be visualized in the editor.\par}
\end{figure}

SBAT's support for Github also enables user management within annotation projects. To accept contributions from multiple annotators in an annotation project, the organizer can create a Github repository for the project and share its access with project members. By doing so, all members can contribute to the same project. Moreover, different branches can be created for different users to create and compare multiple versions. The utilization of Github's collaboration features facilitates in this way management of annotation projects.

\subsection{Browser-Based Application}
SBAT is browser-based, which means one can access its website and then simply start annotating in the browser. But unlike Brat, all logics of SBAT are executed in the browser, so there is no need for software installation or server configuration. Furthermore, by making use of the localStorage of the browser, SBAT can save the user's Personal Access Token once logged in, and automatically log the user in with the same token next time when the site is started in the same browser. This is to simplify the log in process and to allow user to concentrate on annotation tasks. If the user then wishes to log in with a different token, he can utilize the "clear history" function of the browser to clear the previously saved token. Due to its simple construction, it is fast to start and convenient to adapt for different purposes.

\subsection{Flexible Configuration}
Configuration of SBAT is allowed in the config.json file, which makes it easy for users to customize and tailor it to different tasks. For importing files from GitHub, the user is able to enter repository name, owner, branch and file name under the "admin config" attribute in advance, which will allow SBAT to skip the intermediate file selection process in every start [\hyperref[figure3]{Figure 3}] and to directly redirect the user to the editing page of the specified file. This also serves as a solution for importing files that are not located in the first level of the repository but in folders, as it can then not be directly found by user's manual selection. In this case, by
providing the complete path of the file in the config file, e.g. "folder1/folder3/example.json", the user can easily find and open any file that is located in deeper folder structures. \\
\begin{figure}[H]
{\centering
\includegraphics[scale=0.55]{selectFile.png}
\label{figure3}
\caption{Drop-Down List for File Selection}
}
{\setstretch{1.0}\scriptsize The "branch" and "filePath" attributes in "admin config" are optional. If no value is provided, SBAT will prompt user with a drop-down list with available branches or files to select from.\par}
\end{figure}
Details about annotation labels are also defined in the config file. For example, the user can define multiple entity types, their attributes, their children, and the background color for visualization when annotated. They will then appear in the pop-up menu as available options when the user adds a new annotation in the editor [\hyperref[figure1]{Figure 1}]. Besides entity types, there are also event types, event attribute types, relation types, and relation attribute types that can be defined for annotations, each for different annotating tasks.  A more complete example of definition of such types can be found in the config.json file in "example data" folder.


\newpage
\section{Implementation}
\subsection{Annotation UI}
SBAT's annotation editor is rooted in the frontend component from Brat, which is implemented using XHTML, Scalable Vector Graphics (SVG), JavaScript, and jQuery. On top of that, to remove Brat's server-side dependence, the user operations in the editor, such as creating, editing or deleting annotations, are all handled by the browser. This is realized by the LocalAjax class, adapting the LocalAjax function from brat-frontend-editor, which locally simulates AJAX requests to process the data without actually sending real network requests. Other UI functions, such as the user authentication, import and export functions, are also implemented natively in JavaScript. In this way, all components of the user interface are embedded in a HTML webpage and their functionalities are locally executed in the browser.
\subsection{Browser-Based Construction}
As SBAT is a static web app without server communications, it can be hosted on Github Pages for free. It is not mandatory, though, to host the app on the internet in order to use it, as the index.html file can be used as an entry point to start the app locally in the browser. Nonetheless, if the user plans to access it online instead of locally, he can directly upload the source code to his Github repository to host a copy of SBAT on his Github Pages. The link to the hosted website can then be shared with others for online access. Moreover, to cater to different use cases, it is also easy to modify any part of the app -- SBAT replaces the client-side CommonJS modules of BRAT with ECMAScript modules, so all modules are natively loaded in the browser and no module bundler such as webpack is used. This allows direct modification of individual modules without the need for re-bundling.
\subsection{Usage of Github API}
SBAT uses Octokit \footnote{https://esm.sh/octokit@2.1.0, last accessed on September 20th, 2024}, the official SDKs for the Github API, which enables the programmatical implementation of Github operations. The user authentication through Github Personal Access Token creates a new user-authenticated Octokit object, which can then be used to obtain information about branches and file content from GitHub repositories that the authenticated user has access to. Likewise, committing annotations to a Github repository is realized by first obtaining the tree of all files from the last commit in the repository, among the files modifying the changed file or creating a new file, and finally creating a new tree with the updated files and committing it with the REST API endpoint of Octokit. Besides, error handling is implemented for every API call, which catches any potential error and prompts the user with informative error messages and hints.\\
\\
During development, a test Github Personal Access Token is generated under GitHub -> Settings -> Developer Settings -> Personal access tokens. As SBAT only performs the absolutely necessary operations with GitHub authenticated access, which are retrieving list of branches and file names of a repository, obtaining file content, getting information about last commit, and creating new commit, the scope of the Personal Access Token generated for SBAT only needs to cover the repository operations under "Full control of private repositories".
\newpage
\section{Conclusion}
Text annotation is an important task in linguistic studies, and there have been various applications developed to assist the annotation process. They provide editor-like user interfaces, designed for general or specific annotation tasks according to different annotation schemes. Many of them also adopt machine-learning models to further speed up manual annotation or validation, and provide support for user management in annotation projects. However, they mostly requires complicated local or server-side configuration, which increases the cost and time for starting an annotation project.\\
\\
To address that limitation, SBAT aims to provide a simpler and no-server solution for annotation projects. It is an annotation tool for attaching labels to text spans, and is completely browser-based and only requires a browser to function, as its functionalities are all implemented in JavaScript and locally executed. It retains the essential features of BRAT's annotation editor, such as annotation operations and visualizations, while offering integration with Github, enabling and the import and export of annotations from and to user-authenticated Github repositories. Although its usage is currently limited to text span annotation, it can be easily adapted or extended to accomodate the needs of different annotation tasks due to its modular structure and flexible configuration.\\ 








%\begin{figure}[H]
%\centering
%\includegraphics[scale=0.6]{flow_chart.png}
%\caption{Flow chart}
%\end{figure}

%\begin{enumerate}
%\item 
%\item 
%\item 
%\end{enumerate}

\newpage
\bibliography{bibliography}
\bibliographystyle{plainnat}

\end{document}
