\documentclass[12ptm a4paper]{article}
\usepackage[T1]{fontenc}
\usepackage[utf8]{inputenc}
\usepackage{natbib}

\usepackage[margin=1.25in]{geometry}
\usepackage{hyperref} % links
\usepackage{parskip} % proper paragraphs, no indentation
\usepackage{pdfpages} % add non-plagiarism statement

% For showing the month + year only as the date:
\usepackage[en-GB]{datetime2}
\DTMlangsetup{showdayofmonth=false}

\usepackage{xcolor}
\usepackage{blindtext} % for adding the "lorem ipsum" filler sections. delete this once you start writing your thesis.

\usepackage{graphicx}
\graphicspath{{./images/}}
\usepackage{float}

\linespread{1.5}

% This is used in the PDF meta data
\title{SBAT: A Simple Browser-based Annotation Tool with Interactive Visualization and Github Support}
\author{Jia Sheng}
\date{\today}

\begin{document}

% The actual title page of your thesis:
\begin{titlepage}
\begin{center}

\hrule
\vspace{0.6cm}
{\bfseries\LARGE
SBAT: A Simple Browser-Based Annotation Tool with Interative Visualization and Github Support
}\\[1cm]
\hrule
\vspace*{.05\textheight}
 
\begin{minipage}[t]{0.49\textwidth}
\begin{flushleft}
{\large
\textit{Author}\\
Jia Sheng}\\
\href{mailto:jia.sheng@student.uni-tuebingen.de}{\textit{jia.sheng@student.uni-tuebingen.de}}\\
\end{flushleft}
\end{minipage}
\begin{minipage}[t]{0.49\textwidth}
\begin{flushright}
{\large
\textit{Supervisor}\\
Çagri Çöltekin}\\
\href{mailto:ccoltekin@sfs.uni-tuebingen.de}{\textit{ccoltekin@sfs.uni-tuebingen.de}}\\
\end{flushright}
\end{minipage}\\

\vfill

A thesis submitted in partial fulfilment\\
of the requirements for the degree of\\[2mm]
{\large Bachelor of Arts}\\
in\\[1mm]
{\large International Studies in Computational Linguistics}

\vspace*{.1\textheight}

{\large Seminar für Sprachwissenschaft\\
Eberhard Karls Universität Tübingen

\vspace{1em}
\today}
\end{center}
\end{titlepage}

% No page numbering for the abstract, the anti-plagiarism statement and the table of contents.
\pagenumbering{gobble}

%% Uncomment once you want to add the anti-plagiarism statement file
% \newpage
% \includepdf[pages=-]{name-of-anti-plagiarism-statement-file}

\newpage
\tableofcontents
%\listoftables
%\listoffigures
\newpage
\pagenumbering{arabic}

\section{Introduction}
This paper introduces SBAT, a lightweight and easy-to-use web annotation tool. It supports annotation tasks that require attaching labels to text data, such as POS tagging or labeling for named entity recognition. Various apps have been developed for the purpose of text annotation, but they often require a dedicated server or a local installation, which complicates the usage. Therefore, SBAT is designed to be an online annotation tool that is easy to access and simple to use. It is built on the Brat annotation interface, but is completely browser-based and server-free, while preserving most of the key functionalities. Additionally, it includes support for integration with Github's file management system.


\section{Related Work}
Among the annotation tools, Brat \cite{stenetorp-etal-2012-brat} is a powerful scheme-neutral annotation tool with an intuitive user interface. Brat can be used for tasks including but not limited to POS tagging, named entity recognition, semantic role labeling, dependency and verb-frame annotations, and enables user-defined constraints-checking. It provides a high-quality visualization of annotations by its browser-based UI component implemented in XHTML, Scalable Vector Graphics (SVG) and JavaScript. Last but not least, Brat integrats a ML-based semantic class disambiguator that outputs multiple annotations with their possibility estimates, which has been proven to increase the efficiency of annotators in several annotation projects.\\
\\
Despite the success of Brat, it is not the simplest tool to start using. A CGI-capable server is required by its installation, in order to combine all user modifications in real-time with the stored data. That is where brat-frontend-editor \cite{brat-frontend-editor}, a forked project of Brat, differentiates itself from the original project. Brat-frontend-editor is a standalone browser version of Brat that keeps some of its essential editing functions but removes its server-side code. It can be imported as a module in vanilla JavaScript, Angular or React. This project is since 2018 no longer maintained, so many dependencies have broken, and it no longer keeps up with the new updates of the original Brat. However, its transformation of server-side operations into browser-side operations serves as an important reference in the development of SBAT.\\
\\
While Brat serves as a general-purpose annotation tool, more recent tools target specific annotation schemes. UD Annotatrix \cite{tyers-etal-2017-ud}, for example, is specifically designed for annotations of universal dependencies. It accepts several input formats including plain text and CoNNL-U, and offers graphical editing functions. Unlike Brat, it offers a stand-alone module written in JavaScript with locally saved dependencies, which stores imported corpora in localStorage and allows offline usage. Otherwise, there is also a server module to save corpora on the server.\\
\\
Another specialized annotation tool is INCEpTION \cite{klie-etal-2018-inception}, which is used for semantic annotation. One of its characteristics is the extensive adoption of machine-learning models to assist the annotation process. It not only utilizes recommenders based on machine learning models to provide users with suggestions for possible labels, but also provides an active learning mode where user feedback can be used to further improve the quality of those suggestions. It also has a comprehensive user management system, where the admin can create multiple user accounts for the project and assign them different roles. In terms of the software itself, INCEpTION is a Java application with Spring Boot backend and a web user interface, so its usage requires the installation of Java as well as the software.\\
\\
Lastly, the previous version of SBAT \cite{SBAT} serves as one of the foundations for the current version. The previous version shares the same goal: to create a simple but functional annotation tool that operates entirely in the browser, without any server-side configuration. It is implemented using HTML and JavaScript, and can be used to attach user-defined labels to paragraphs. It also allows the local import and export of annotation files. Building on that, the current version addresses the shortcomings of the previous one by adding functionalities such as span annotations, annotation visualization, and Github support.
\section{Features}
\subsection{Interactive User Interface}
Based on Brat, the annotation UI of SBAT is responsive, interactive and intuitive to use. To add an annotation, one can simply double-click or select a span with his mouse, and then select corresponding label in the pop-up window. All added labels will be visualized with tags above the annotated text, and different tags differentiate with each other by the user-defined color. And if the user has added comments, they will be shown with a mouse hover. The size and display of the interface can be adjusted by the option button, and the labels and their style are defined in the JSON config file. Besides the main component for annotating text, there is also a step-by-step guide in the beginning for github authentication as well as the export function, all with clear instructions.
\subsection{Browser-Based Application}
SBAT is browser-based, which means one can simply open the website and start annotating in the browser. But unlike Brat, all logics of SBAT are completely executed in the browser, there is no need of software installation or server configuration. One can either store its front-end files in a Github repository to access it via Github pages online, or download the those files to start it locally in the browser. Due to its simple structure, it is fast to run and easy to adapt for different purposes.
\subsection{Github Support}
SBAT can not only be hosted on Github Pages, but also supports import files from or export annotations to Github repositories. To do so, the user only has to enter his Github Personal Access Token (information about the repository name as well as the its owner's user name are entered in advance under "admin config" in the JSON config file), and can then select the github branch as well as the file to open. For importing file, all file formats are supported, and they will be treated as txt file, i.e. the whole content will be displayed as annotation-ready text in the interface. The only exception is JSON file which has been previously exported from SBAT (which has attributes like text, tokens, entities...) -- Their content will be parsed and the annotations saved in the file will be visualized. To export annotations, the user can optionally leave a commit message and click the "commit" button, a JSON file with original texts and annotations will be commited to the same Github repository from which the file was originally imported. By making use of Github commit, the versions of annotations can be well documented and managed.
\subsection{Project Management Support}
SBAT's support for Github also provides possibility for user management in annotation projects. For instance, the admin can create a Github repository for an annotation project and add the repository and owner information in SBAT's config file. By sharing access to this repository with different users, they can contribute to the same project at the same time. Moreover, different branches can be created for different users to create and compare multiple versions. This utilization of Github's file management system facilitates the colloboration of multiple annotators in an annotation project.
\section{Implementation}
\subsection{Annotation Interface}
SBAT's annotation interface is rooted in the frontend component from Brat, which is implemented using XHTML, Scalable Vector Graphics (SVG), JavaScript, and jQuery. On top of that, to remove the dependance on a server, all user actions such as creating, editing or deleting annotations are handled by the browser. This is realized by the LocalAjax class, adapting the LocalAjax function from brat-frontend-editor. It locally simulates AJAX requests to process the data without actually sending real network requests. Other UI elements, such as the login buttons, are also created by HTML and Asynchronous JavaScript. In this way, the complete user interface is embedded in a HTML webpage and all functions are locally executed in the browser.
\subsection{Browser-Based Construction}
As SBAT is a static web app without server communications, it can be hosted on Github Pages for free. But it is not mandatory to host the app in order to use it, as the index.js file can be used as an entry point to start the app locally in the browser. If the users want to access it online instead of locally, the source code can be directly uploaded to a Github repository to host a copy of SBAT on Github Pages. Moreover, to cater to different use cases, it is also easy to modify any part of the app -- SBAT uses ECMAScript Modules instead of CommonJS, so all modules are natively loaded in browser and no module bundler such as webpack is used. This allows direct modification of individual modules without the need of re-bundling.
\subsection{Usage of Github API}
SBAT uses Octokit \cite{octokit.js}, the official SDKs for the Github API, which enables the programmatical implementation of Github operations. The user authentication through Github Personal Access Token enables creating a new user-authenticated Octokit object and obtaining information about branches and file content from the configured repository. Likewise, creating a new commit and pushing to github repository is realized by first obtaining tree of all files from last commit, modifying the changed file or creating a new file, and the creating the new tree and commit it with the REST API endpoint of octokit.

\section{Conclusion}
SBAT is an annotation tool that can be used to attach labels to linguistic data. It is completely browser-based and only requires a browser to be executed. Furthermore, it can be combined well with github to import from and export annotations to user's own github repositories. It aims to provide a simpler and no-server solution for annotation projects.\\ 




%\begin{figure}[H]
%\centering
%\includegraphics[scale=0.2]{rag.png}
%\caption{Retrieval Augmented Generation}
%Source: https://techcommunity.microsoft.com/t5/azure-ai-services-blog/revolutionize-your-enterprise-data-with-chatgpt-next-gen-apps-w/ba-p/3762087
%\end{figure}



%\begin{figure}[H]
%\centering
%\includegraphics[scale=0.6]{flow_chart.png}
%\caption{Flow chart}
%\end{figure}

%\begin{enumerate}
%\item 
%\item 
%\item 
%\end{enumerate}


\bibliography{bibliography}
\bibliographystyle{plain}

\end{document}
