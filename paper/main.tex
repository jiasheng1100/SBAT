\documentclass[a4paper]{article}
\usepackage[T1]{fontenc}
\usepackage[utf8]{inputenc}
\usepackage{natbib}

\usepackage[margin=1.25in]{geometry}
\usepackage{hyperref} % links
\usepackage{parskip} % proper paragraphs, no indentation
\usepackage{pdfpages} % add non-plagiarism statement

% For showing the month + year only as the date:
\usepackage[en-GB]{datetime2}
\DTMlangsetup{showdayofmonth=false}

\usepackage{xcolor}
\usepackage{blindtext} % for adding the "lorem ipsum" filler sections. delete this once you start writing your thesis.

\usepackage{graphicx}
\graphicspath{{./images/}}
\usepackage{float}

% This is used in the PDF meta data
\title{SBAT: A Simple Browser-based Annotation Tool with Interactive Visualization and Github Support}
\author{Jia Sheng}
\date{\today}

\begin{document}

% The actual title page of your thesis:
\begin{titlepage}
\begin{center}

\hrule
\vspace{0.6cm}
{\bfseries\LARGE
SBAT: A Simple Browser-based Annotation Tool with Interative Visualization and Github Support
}\\[1cm]
\hrule
\vspace*{.05\textheight}
 
\begin{minipage}[t]{0.49\textwidth}
\begin{flushleft}
{\large
\textit{Author}\\
Jia Sheng}\\
\href{mailto:jia.sheng@student.uni-tuebingen.de}{\textit{jia.sheng@student.uni-tuebingen.de}}\\
\end{flushleft}
\end{minipage}
\begin{minipage}[t]{0.49\textwidth}
\begin{flushright}
{\large
\textit{Supervisor}\\
Çagri Çöltekin}\\
\href{mailto:ccoltekin@sfs.uni-tuebingen.de}{\textit{ccoltekin@sfs.uni-tuebingen.de}}\\
\end{flushright}
\end{minipage}\\

\vfill

A thesis submitted in partial fulfilment\\
of the requirements for the degree of\\[2mm]
{\large Bachelor of Arts}\\
in\\[1mm]
{\large International Studies in Computational Linguistics}

\vspace*{.1\textheight}

{\large Seminar für Sprachwissenschaft\\
Eberhard Karls Universität Tübingen

\vspace{1em}
\today}
\end{center}
\end{titlepage}

% No page numbering for the abstract, the anti-plagiarism statement and the table of contents.
\pagenumbering{gobble}

%% Uncomment once you want to add the anti-plagiarism statement file
% \newpage
% \includepdf[pages=-]{name-of-anti-plagiarism-statement-file}

\newpage
\tableofcontents
%\listoftables
%\listoffigures
\newpage

\pagenumbering{arabic}

\section{Introduction}
This paper introduces a SBAT, a light-weighted and easy-to-use web annotation tool. It supports text annotation tasks that require attachment of labels, such as POS tagging or labeling for named entity recognition. There have been various apps developed for the purpose of linguistic annotation, but they often require a dedicated server or a local installation, which complicates the usage. Basing on the annotation interface of Brat, SBAT is developed for easier start of text annotation while retaining the most essential functions. It is completely browser-based, without the need of any server, and can be well combined with Github's file management function.


\section{Related Work}
Among the annotation tools, Brat \cite{stenetorp-etal-2012-brat} is a powerful scheme-neutral annotation tool with intuitive and user-friendly interfaces. And it is extremely important for this paper as SBAT is developed upon its UI component. Brat can be used for tasks including but not limited to POS tagging, named entity recognition, semantic role labeling, dependancy and verb frame annotations, and enables user-defined contraints-checking. Besides, it provides a high-quality visualization of annotations by its browser-based UI component implemented in XHTML, Scalable Vector Graphics (SVG) and JavaScript. Last but not the least, Brat integrats a ML-based semantic class disambiguator that outputs multiple annotations with their posibility estimates, which was proven to increase the efficiency of annotators.\\
\\
Although Brat is a very capable annotation tool, it is not necessarily the simpliest to use. A CGI-capable server is required by its installation, which will combine all user modifications real-time with the stored data. That is where brat-frontend-editor, a forked project of Brat, differentiates itself with the original project. Brat-frontend-editor \cite{brat-frontend-editor} is a standalone browser version of Brat that do not require a server. This project is since 2018 no longer maintained, however, its transformation of server-side operations into browser-only operations serves as important reference in the development of SBAT.\\
\\
While Brat serves as a general-purpose annotation tool, more recent tools target at more specific annotation schemes. UD Annotatrix \cite{tyers-etal-2017-ud}, for example, is specifically designed for annotations of universal dependencies. It accepts several input formats including plain text and CoNNL-U, and offers user Graphical editing function. Unlike Brat, it offers a stand-alone module, which is written in JavaScript with locally saved dependencies and stores imported corpora in localStorage, allowing offline usage. But there is also a server module to save corpora on server.\\
\\
Another specific-purpose annotation tool is INCEpTION \cite{klie-etal-2018-inception}, which is used for semantic annotation. What differentiates it from the previously-introduced tools is its wide adoption of machine learning models to assist annotation process. It not only utilizes recommenders based on machine learning models to provide users with suggestions for possible labels, but also has an active learning mode where user's feedback can be used to further improve the quality of those suggestions. It also has a comprehensive user management system, where admin can create multiple user accounts for the project and assign them different roles. In terms of the software itself, INCEpTION is a Java application with Spring Boot backend and a web user interface, so its usage requires installation of Java as well as the software.\\
\\
Lastly, the initial version of SBAT \cite{SBAT} serves as one of the basis of the current SBAT version. The development of this previous version has the same goal, which is to achieve a simple but functional annotation tool that is completely browser-based, without any server-side configuration. It is implemented using HTML and JavaScript, and can be used to define and attach labels to paragraphs, as well as import and export annotation files.
\section{Features}
\begin{enumerate}
\item Interactive Annotation Interface
The annotation interface of SBAT is completely based on that of BRAT, which is interactive, intuitive and easy-to-use. To add an annotation, one can simply double-click a word or select a span with the mouse, and then select the labels in the pop-up window. The size and display of the interface can be adjusted by the option button, and the labels and their corresponding color can be defined in the JSON config file.
\item web interface, no need to register
SBAT is browser-based, which means one can simply open the website and start annotating in the browser. But unlike Brat, all logics of SBAT are completely executed in the browser, so there is no need to configure a server or install any other software.
\item import and export annotations through ui from github.
Another advantage of SBAT is its support for github. The user only has to enter his own Github Personal Access Token, and can then import files to annotate from his github repositories and export his annotations by making a commit in the same repository. By making use of github commit, the versions of annotations can be well documented and managed. The annotations are exported to a JSON file with attributes like text, tokens, entites.... For file import, all file types are supported, but only annotations in JSON file with these specific attributes can be visualized in the annotation interface, other file types will be simply treated as txt file, i.e. the complete content will be displayed as pure text which can be annotated.
\item potential for collaboration in annotation project
The usage of github repositories to store annotations also enables colloboration and project management. There is a admin config attribute in the JSON config file, which can be modified by the admin to specify the common repository to work on. Then, by sharing the repository to all team members, a team will be able to work on the same annotation project. Different branches can also be created for different members to create and compare multiple versions of annotations.
\end{enumerate}
\section{Implementation}
\begin{enumerate}
\item annotation ui from brat
The annotation user interface is made with the frontend UI component from Brat, which is implemented using XHTML, Scalable Vector Graphics (SVG), JavaScript, and jQuery. Moreover, to get rid of the dependancy on a server, all user actions such as creating, editing or deleting annotations are handled by the browser. This is realized by the LocalAjax class, which is written based on the LocalAjax function in brat-frontend-editor and is a local simulation of AJAX requests processing the data without actually sending real network requests. Other UI elements, such as the login buttons, are also implemented using HTML and Asynchronous JavaScript.
\item static web app hosted on github pages, no need of server, no server-side configuration, light-weighted
SBAT is a static web app that is hosted on github pages, and as there is no need of a server or any server-side configuration, anyone who have downloaded the repository can simply change the configurations and then host it on their own github pages. If it is not an annotation project which requires collaboration of many members, SBAT can also simply be used locally, by starting the index.js file in the local browser.
\item github API for reading and writing annotation files
SBAT uses Octokit, the official SDKs for the Github API.
octokit.js v4.0.2 [Computer software]. (2024). Retrieved from https://github.com/octokit/octokit.js
The user authentication through Github Personal Access Token enables creating a new user-authenticated Octokit object and obtaining information about branches and file content from the configured repository.  Likewise, creating a new commit and pushing to github repository is realized by first obtaining tree of all files from last commit, modifying the changed file or creating a new file, and the creating the new tree and commit with the octokit REST API endpoint.
\end{enumerate}

\section{Conclusion}
SBAT is an annotation tool that can be used to attach annotation labels to linguistic data. It is completely browser-based and only requires a browser to be executed. Furthermore, it can be combined well with github to import from and export annotations to user's own github repositories. It aims to provide a more simplified and no-server solution for annotation projects.\\ 




%\begin{figure}[H]
%\centering
%\includegraphics[scale=0.2]{rag.png}
%\caption{Retrieval Augmented Generation}
%Source: https://techcommunity.microsoft.com/t5/azure-ai-services-blog/revolutionize-your-enterprise-data-with-chatgpt-next-gen-apps-w/ba-p/3762087
%\end{figure}



%\begin{figure}[H]
%\centering
%\includegraphics[scale=0.6]{flow_chart.png}
%\caption{Flow chart}
%\end{figure}

%\begin{enumerate}
%\item 
%\item 
%\item 
%\end{enumerate}


\bibliography{bibliography}
\bibliographystyle{plain}

\end{document}
